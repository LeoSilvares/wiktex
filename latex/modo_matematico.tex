%meta-autor LeoSilvares
%meta-versao 1
%meta-data 02/Aug/2021 13:10:17
%conteudo

%meta-autor LeoSilvares
%meta-versao 0
%meta-data 02/Ago/2021 10:00:00
%conteudo

\newcommand{\com}[1]{\textbf{\backslash#1}}
\newcommand{\comp}[2]{\textbf{\backslash#1\{#2\}}}
\newcommand{\compp}[3]{\textbf{\backslash#1\{#2\}\{#3\}}}
\newcommand{\comop}[3]{\textbf{\backslash#1[#2]\{#3\}}}

\subsection{Modo matemático}

O modo matemático em linha pode ser nas formas \com{(}x^2\com{)} ou \textbf{\$}x^2\textbf{\$}. O modo matemático em uma nova linha pode ser nas formas \com{[}x^2\com{]} ou \textbf{\$\$}x^2\textbf{\$\$}.\\

Também são aceitos diversos ambientes matemáticos, veja \href{#am}{abaixo}.\\

A renderização das equações é feita pelo \href{mathjax.org}{MathJax}.

