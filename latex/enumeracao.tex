%meta-autor LeoSilvares
%meta-versao 2
%meta-data 13/Aug/2021 18:17:43
%conteudo

%meta-autor LeoSilvares
%meta-versao 0
%meta-data 02/Ago/2021 10:00:00
%conteudo

\newcommand{\com}[1]{\textbf{\backslash#1}}
\newcommand{\comp}[2]{\textbf{\backslash#1\{#2\}}}
\newcommand{\compp}[3]{\textbf{\backslash#1\{#2\}\{#3\}}}
\newcommand{\comop}[3]{\textbf{\backslash#1[#2]\{#3\}}}

\section{Ambientes de enumeração}

\begin{itemize}
    \item Enumeração padrão: o código
    \begin{verbatim}
        \begin{enumerate}
            \item Item
            \item Outro item
        \end{enumerate}\end{verbatim}
    produzirá:
\end{itemize}
    \begin{enumerate}
        \item Item
        \item Outro item
    \end{enumerate}

\begin{itemize}
    \item Enumeração com estilo: o código
    \begin{verbatim}
        \begin{enumerate}[(a)]
            \item Item
            \item Outro item
        \end{enumerate}\end{verbatim}
    produzirá:
\end{itemize}
    \begin{enumerate}[(a)]
        \item Item
        \item Outro item
    \end{enumerate}
    Pode-se utilizar no estilo os caracteres '(', ')', '-', '.' e ':'. A numeração pode ser com \textbf{1} (números), \textbf{I} (romanos maiúsculos), \textbf{i} (romanos minúsculos), \textbf{a} (letras latinas minúsculas) ou \textbf{A} (letras latinas maiúsculas).

\begin{itemize}
    \item Itemização com estilo: o código
    \begin{verbatim}
        \begin{itemize}
            \item Item
            \item Outro item
        \end{enumerate}\end{verbatim}
    produzirá:
\end{itemize}
    \begin{minipage}{80px}$\;$\end{minipage}\begin{minipage}{400px}
    \begin{itemize}
        \item Item
        \item Outro item
    \end{itemize}
    \end{minipage}
    
\begin{itemize}
    \item Itemização com colunas: o código
    \begin{verbatim}
        \begin{itemize-cols}{200px}
            \item Item 1
            \item Item 2
            \item Item 3
            \item Item 4
            \item Item 5
            \item Item 6
        \end{itemize-cols}
    \end{verbatim}
    produzirá uma lista em que os elementos podem ficar lado a lado, em caixas de 200px de largura, que se distribuirão de acordo com a largura da página:
\end{itemize}
    \begin{itemize-cols}{200px}
        \item Item 1
        \item Item 2
        \item Item 3
        \item Item 4
        \item Item 5
        \item Item 6
    \end{itemize-cols}

     Redimensione a janela do navegador para ver o efeito.