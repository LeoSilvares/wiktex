%meta-autor LeoSilvares
%meta-versao 4
%meta-data 23/Aug/2021 12:00:54
%conteudo

%meta-autor LeoSilvares
%meta-versao 0
%meta-data 02/Ago/2021 10:00:00
%conteudo

\newcommand{\com}[1]{\textbf{\backslash#1}}
\newcommand{\comp}[2]{\textbf{\backslash#1\{#2\}}}
\newcommand{\compp}[3]{\textbf{\backslash#1\{#2\}\{#3\}}}
\newcommand{\comop}[3]{\textbf{\backslash#1[#2]\{#3\}}}

\subsection{Ambientes específicos}

Os seguintes ambientes são suportados:

\begin{itemize}
    \item HTML: o código abaixo produzirá uma saída HTML exatamente como for escrita.
        \begin{verbatim}
            \begin{html}
                <a href='google.com'><div style='display: inline-block; width:fit-content; border: 2px solid black;'>Google</div></a>
            \end{html}\end{verbatim}
        produzirá:
        \begin{html}
            <a href='google.com'><div style='display: inline-block; width:fit-content; border: 2px solid black;'>Google</div></a>
        \end{html}
    
    \item Verbatim: o ambiente verbatim faz com que o texto em seu interior seja reproduzido identicamente na saída, sem qualquer interpretação LaTeX.
    
        \begin{verbatim}
            \begin{‎verbatim}
                \section{Funções reais}
            \end{‎verbatim}\end{verbatim}
        produzirá:
        
        \begin{verbatim}
            \section{Funções reais}\end{verbatim}
    
    \item Comentários no código: tudo o que for escrito no ambiente comment será ignorado pelo interpretador:
    
    \begin{verbatim}
        Isto será interpretado
        \begin{comment}
            Isto não será interpretado. x^2+1, \ref{ref}    
        \end{comment}
        Isto será.
    \end{verbatim}
        produzirá:
        
        Isto será interpretado
        \begin{comment}
            Isto não será interpretado, mesmo com erros x^2^2 ou comandos como \ref{ref}.    
        \end{comment}
        Isto será.
    
    
\end{itemize}
    
		