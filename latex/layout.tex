%meta-autor LeoSilvares
%meta-versao 1
%meta-data 02/Aug/2021 13:11:04
%conteudo

%meta-autor LeoSilvares
%meta-versao 0
%meta-data 02/Ago/2021 10:00:00
%conteudo

\newcommand{\com}[1]{\textbf{\backslash#1}}
\newcommand{\comp}[2]{\textbf{\backslash#1\{#2\}}}
\newcommand{\compp}[3]{\textbf{\backslash#1\{#2\}\{#3\}}}
\newcommand{\comop}[3]{\textbf{\backslash#1[#2]\{#3\}}}

\section{Ambientes de layout}

\begin{itemize}
    \item Minipágina: o código
    \begin{verbatim}
        \begin{minipage}{300}
            Exemplo de minipágina.
            \begin{theorem}
                Teorema em minipágina
            \end{theorem}
        \end{minipage}
        \hspace{50}
        \begin{minipage}{400}
            Outro exemplo de minipágina.
            \begin{example}
                Exemplo em minipágina
            \end{example}
        \end{minipage}\end{verbatim}
    produzirá duas minipáginas lado a lado, com larguras 300 e 400 pixels, respectivamente:
    \begin{minipage}{300}
        Exemplo de minipágina.
        \begin{theorem}
            Teorema em minipágina
        \end{theorem}
    \end{minipage}
    \hspace{50}
    \begin{minipage}{400}
        Outro exemplo de minipágina.
        \begin{example}
            Exemplo em minipágina
        \end{example}
    \end{minipage}

    Repare que o \comp{hspace}{50} produz um espaçamento de 50 pixels entre as minipáginas.
\end{itemize}
