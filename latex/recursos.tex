%meta-autor LeoSilvares
%meta-versao 1
%meta-data 02/Aug/2021 13:10:06
%conteudo

%meta-autor LeoSilvares
%meta-versao 0
%meta-data 02/Ago/2021 10:00:00
%conteudo

\newcommand{\com}[1]{\textbf{\backslash#1}}
\newcommand{\comp}[2]{\textbf{\backslash#1\{#2\}}}
\newcommand{\compp}[3]{\textbf{\backslash#1\{#2\}\{#3\}}}
\newcommand{\comop}[3]{\textbf{\backslash#1[#2]\{#3\}}}

\subsection{Recursos multimída}

\begin{itemize}
    \item \comop{includegraphics}{argumentos opcionais}{imagem}: insere uma imagem do repositório de mídia, que pode ser acessado no botão "Mídia" da página de edição. Os argumentos opcionais devem ser na forma 
    \begin{center}
        \texttt{[Opção1=valor, ..., OpçãoN=valor]}
    \end{center}
    onde, cada OpçãoX corresponde a um campo de estilo CSS ho html (exemplos: width [largura], height [altura])
    \item \compp{youtube}{id do vídeo}{texto explicativo}: Insere uma caixa com texto explicativo que, quando clicada, abre um vídeo do YouTube. O id do vídeo são as letras/números que aparecem no final do link do vídeo. Exemplo de vídeo:
    \youtube{JJ-Y65hFIrA}{Assista ao vídeo "Expressões do Primeiro Grau", que cobre parte do conteúdo desta página}
    criado com o comando
    \begin{center}
        \texttt{\com{youtube}\{JJ-Y65hFIrA\}\{Assista ao vídeo "Expressões do Primeiro Grau", que cobre parte do conteúdo desta página\}}
    \end{center}
    
    \item \comop{geogebra}{argumentos opcionais}{imagem}: insere uma imagem do repositório de mídia, que pode ser acessado no botão "Mídia" da página de edição. Os argumentos opcionais devem ser na forma 
    \begin{center}
        \texttt{[Opção1=valor, ..., OpçãoN=valor]}
    \end{center}
    onde, cada OpçãoX pode corresponder a a um campo de estilo CSS ho html (exemplos: width [largura], height [altura]), ou a um parâmetro do GeoGebra, se inicado com "G_" (por exemplo, "G_width=300" passará o parâmetro width como 300 para o applet).
\end{itemize}

